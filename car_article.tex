\documentclass[a4paper,11pt]{article}
\usepackage{CAR}
\usepackage[T1]{fontenc}
%\usepackage[latin1]{inputenc}
\usepackage{amsfonts}
\usepackage{amsmath}
\usepackage{amssymb}
\usepackage{multicol}
\usepackage{enumitem}

\newenvironment{my_enumerate}{
\begin{enumerate}
  \setlength{\itemsep}{0.2pt}
  \setlength{\parskip}{0pt}
  \setlength{\parsep}{0pt}}{\end{enumerate}
}

\begin{document}
\setcounter{footnote}{0}
\setcounter{figure}{0}

%%%%%%%%%%%%%%%%%%%%%%%%%%%%%%%%%%%%%%%%%%%%%%%%%%%%%%%%%%%%%%%%%%%%%%%%%%%%%%%%
% FOR THE EDITOR
%%%%%%%%%%%%%%%%%%%%%%%%%%%%%%%%%%%%%%%%%%%%%%%%%%%%%%%%%%%%%%%%%%%%%%%%%%%%%%%%

\Abschnitt
% Name der Rubrik
{Rubrik}
% Name der Rubrik (Inhaltsverzeichnis)
{Rubrik}
% Label fuer die Rubrik
{rubrik}

\vspace{3mm}

%%%%%%%%%%%%%%%%%%%%%%%%%%%%%%%%%%%%%%%%%%%%%%%%%%%%%%%%%%%%%%%%%%%%%%%%%%%%%%%%
% FOR THE AUTHORS
%%%%%%%%%%%%%%%%%%%%%%%%%%%%%%%%%%%%%%%%%%%%%%%%%%%%%%%%%%%%%%%%%%%%%%%%%%%%%%%%

\Aufsatz
% Title
{Post-Quantum Secure Cryptographic Algorithms}
% Short title for the table of contents
{Overview of Developments 2017/2018}
% Authors
{Dipl. Math. Xenia Bogomolec, Dr. Jochen Gerhard}
% Label of authors' names
{NameAuthor}
% Names and adresses
{Xenia Bogomolec, University of ...\\ Dr. Jochen Gerhard, BearingPoint Software Solutions GmbH}
% Images of authors
{./xenia.jpg,./jochen.jpg}
% E-Mails
{m.name@email.de,jochen.gerhard@bearingpoint.com}



%%%%%%%%%%%%%%%%%%%%%%%%%%%%%%%%%%%%%%%%%%%%%%%%%%%%%%%%%%%%%%%%%%%%%%%%%%%%%%%%
% IF THE ARTICLE IS WRITTEN IN ENGLISH, THEN UNCOMMENT
% THE NEXT LINE AND ANOTHER LINE AT THE END OF THIS FILE
\begin{otherlanguage}{english}
%%%%%%%%%%%%%%%%%%%%%%%%%%%%%%%%%%%%%%%%%%%%%%%%%%%%%%%%%%%%%%%%%%%%%%%%%%%%%%%%

\vspace{3mm}
\begin{multicols}{2}
\noindent

%%%%%%%%%%%%%%%%%%%%%%%%%%%%%%%%%%%%%%%%%%%%%%%%%%%%%%%%%%%%%%%%%%%%%%%%%%%%%%%%
% WRITE YOUR ARTICLE BELOW using \Ueberschrift \Ueberschriftu \begin{figurehead}
%%%%%%%%%%%%%%%%%%%%%%%%%%%%%%%%%%%%%%%%%%%%%%%%%%%%%%%%%%%%%%%%%%%%%%%%%%%%%%%%

% The command \Ueberschrift{Title}{label} produces a headline and new section
\Ueberschrift{Introduction}{intro}

% The command \Ueberschriftu{Title} produces a subsection
\Ueberschriftu{A subsection}

%%% For figures
%\begin{figurehere}
%  \centering
%  \includegraphics[width=\columnwidth]{Abbildung}
%  \caption{First figure of this article.\label{abb_1}}
%\end{figurehere}


%
% C
% O
% N
% T
% E
% N
% T
%
% H
% E
% R
% E
%


%%%%%%%%%%%%%%%%%%%%%%%%%%%%%%%%%%%%%%%%%%%%%%%%%%%%%%%%%%%%%%%%%%%%%%%%%%%%%%%%
% Literature
%%%%%%%%%%%%%%%%%%%%%%%%%%%%%%%%%%%%%%%%%%%%%%%%%%%%%%%%%%%%%%%%%%%%%%%%%%%%%%%%

\begin{thebibliography}{1}
\itemsep=0cm plus 0pt minus 0pt


% Macro for an entry in the bibliography
\bibitem
% label
{B1}
% authors
Max~H.\ Mustermann und Hans~M.\ N.N.
% title
\newblock Example of a title.
% Journal
\newblock {\em Computeralgebra-Rundbrief}, 49:0--1, Oktober 2011.



\end{thebibliography}

%%%%%%%%%%%%%%%%%%%%%%%%%%%%%%%%%%%%%%%%%%%%%%%%%%%%%%%%%%%%%%%%%%%%%%%%%%%%%%%%

\end{multicols}



%%% IF ENGLISH:
\end{otherlanguage}
%%%

\end{document}
